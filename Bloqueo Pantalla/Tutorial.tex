%%%%%%%%%%%%%%%%%%%%%%%%%%%%%%%%%%%%%%%%%
% Short Sectioned Assignment LaTeX Template Version 1.0 (5/5/12)
% This template has been downloaded from: http://www.LaTeXTemplates.com
% Original author:  Frits Wenneker (http://www.howtotex.com)
% License: CC BY-NC-SA 3.0 (http://creativecommons.org/licenses/by-nc-sa/3.0/)
%%%%%%%%%%%%%%%%%%%%%%%%%%%%%%%%%%%%%%%%%

%----------------------------------------------------------------------------------------
%	PACKAGES AND OTHER DOCUMENT CONFIGURATIONS
%----------------------------------------------------------------------------------------

\documentclass[paper=a4, fontsize=11pt]{scrartcl} % A4 paper and 11pt font size

% ---- Entrada y salida de texto -----

\usepackage[T1]{fontenc} % Use 8-bit encoding that has 256 glyphs
\usepackage[utf8]{inputenc}
%\usepackage{fourier} % Use the Adobe Utopia font for the document - comment this line to return to the LaTeX default

% ---- Idioma --------

\usepackage[spanish, es-tabla]{babel} % Selecciona el español para palabras introducidas automáticamente, p.ej. "septiembre" en la fecha y especifica que se use la palabra Tabla en vez de Cuadro

% ---- Otros paquetes ----

\usepackage{amsmath,amsfonts,amsthm} % Math packages
%\usepackage{graphics,graphicx, floatrow} %para incluir imágenes y notas en las imágenes
\usepackage{graphics,graphicx, float} %para incluir imágenes y colocarlas

% Para hacer tablas comlejas
%\usepackage{multirow}
%\usepackage{threeparttable}

%\usepackage{sectsty} % Allows customizing section commands
%\allsectionsfont{\centering \normalfont\scshape} % Make all sections centered, the default font and small caps

\usepackage{fancyhdr} % Custom headers and footers
\usepackage{url}
\usepackage{hyperref}
\pagestyle{fancyplain} % Makes all pages in the document conform to the custom headers and footers
\fancyhead{} % No page header - if you want one, create it in the same way as the footers below
\fancyfoot[L]{} % Empty left footer
\fancyfoot[C]{} % Empty center footer
\fancyfoot[R]{\thepage} % Page numbering for right footer
\renewcommand{\headrulewidth}{0pt} % Remove header underlines
\renewcommand{\footrulewidth}{0pt} % Remove footer underlines
\setlength{\headheight}{13.6pt} % Customize the height of the header

\numberwithin{equation}{section} % Number equations within sections (i.e. 1.1, 1.2, 2.1, 2.2 instead of 1, 2, 3, 4)
\numberwithin{figure}{section} % Number figures within sections (i.e. 1.1, 1.2, 2.1, 2.2 instead of 1, 2, 3, 4)
\numberwithin{table}{section} % Number tables within sections (i.e. 1.1, 1.2, 2.1, 2.2 instead of 1, 2, 3, 4)

\setlength\parindent{0pt} % Removes all indentation from paragraphs - comment this line for an assignment with lots of text

\newcommand{\horrule}[1]{\rule{\linewidth}{#1}} % Create horizontal rule command with 1 argument of height




%----------------------------------------------------------------------------------------
%	TÍTULO Y DATOS DEL ALUMNO
%----------------------------------------------------------------------------------------

\title{	
\normalfont \normalsize 
\textsc{{\bf Nuevos Paradigmas de Interacción (2015-2016)} \\ Grado en Ingeniería Informática \\ Universidad de Granada} \\ [25pt] % Your university, school and/or department name(s)
\horrule{0.5pt} \\[0.4cm] % Thin top horizontal rule
\huge Tutorial appMoveMvl \\ % The assignment title
\horrule{2pt} \\[0.5cm] % Thick bottom horizontal rule
}

\author{Pedro Antonio Ruiz Cuesta\\
Ignacio Martín Requena} % Nombre y apellidos

\date{\normalsize\today} % Incluye la fecha actual

%----------------------------------------------------------------------------------------
% DOCUMENTO
%----------------------------------------------------------------------------------------

\usepackage{graphicx}
\usepackage{listings}
\usepackage{color}
\definecolor{gray97}{gray}{.97}
\definecolor{gray75}{gray}{.75}
\definecolor{gray45}{gray}{.45}

\lstset{ frame=Ltb,
     framerule=0pt,
     aboveskip=0.1cm,
     framextopmargin=3pt,
     framexbottommargin=3pt,
     framexleftmargin=0.2cm,
     framesep=0pt,
     rulesep=.2pt,
     backgroundcolor=\color{gray97},
     rulesepcolor=\color{black},
     %
     stringstyle=\ttfamily,
     showstringspaces = false,
     basicstyle=\small\ttfamily,
     commentstyle=\color{gray45},
     keywordstyle=\bfseries,
     %
     numbers=left,
     numbersep=15pt,
     numberstyle=\tiny,
     numberfirstline = false,
     breaklines=true,
   }
 


\lstdefinestyle{consola}
   {basicstyle=\scriptsize\bf\ttfamily,
    backgroundcolor=\color{gray75},
   }
 
\lstdefinestyle{C}
   {language=C,
   }


\begin{document}

\maketitle % Muestra el Título

\newpage %inserta un salto de página

\tableofcontents % para generar el índice de contenidos

\listoffigures

\newpage


%----------------------------------------------------------------------------------------
%	Introducción
%----------------------------------------------------------------------------------------
\section{Introducción}
En este tutorial veremos como hemos desarrollado una app para detectar cuando el sensor de proximidad detecta que un objeto está próximo al terminal.


%----------------------------------------------------------------------------------------
%	Descripción
%----------------------------------------------------------------------------------------
\section{Descripción}
Esta aplicación consta de tres activitis el bloqueo de pantalla (la principal), el bloqueo pantalla eventos (recive eventos y los pone en los listeners adecuados), el bloqueo panralla llamada (para gestionar el desbloqueo del terminal cuando se produce una llamada) y el bloqueo pantalla permisos (para gestionar los permisos que se le dan a la aplicación).

%----------------------------------------------------------------------------------------
%	Desarrollo
%----------------------------------------------------------------------------------------
\section{Desarrollo}
Para el desarrollo de nuestra app utilizaremos el IDE Android Studio.\\

Para la parte de la lectura del código hemos utilizado la librería SensorEventListener que nos permite acceder a la información recogida por el sensor de posición.\\

\subsection{Iniciando el proyecto}
Lo primero que vamos a hacer es crear un nuevo proyecto en android studio, elegiremos que nos cree un proyecto vacío con el nombre Main Activity.


Una vez creado el proyecto solo quedaría modelar la interfaz de usuario y determinar el código fuente de nuestro programa.

\subsection{Determinación de la proximidad de un objeto al terminal}
El código de nuestra app que determina si el terminal esta boca abajo es:

\begin{lstlisting}[language=Java]
/**
* Metodo que evalua el estado del sensor
* @param event
*/
@Override
public void onSensorChanged(SensorEvent event) {
  if (event.values[0] == 0)
    turnScreenOffAndExit();
}
\end{lstlisting}

Si el valor de event.values[0] es 0 quiere decir que el sensor de proximidad ha detectado que hay un objeto, así que llamamos a la función para bloquear el terminal.

\subsection{Bloqueo del terminal}

\begin{lstlisting}[language=Java]
/**
* Metodo principal que apaga la pantalla y sale de la app
*/
private void turnScreenOffAndExit() {
  // Bloqueamos la mantalla
  turnScreenOff(getApplicationContext());
	
  // Hacemos vibrar el movil
  ((Vibrator) getSystemService(Context.VIBRATOR_SERVICE)).vibrate(50);
	
  final Activity activity = this;
  Thread t = new Thread() {
    public void run() {
      try {
        sleep(500);
      } catch (InterruptedException e) { }
      activity.finish();
    }
  };
  t.start();
}
\end{lstlisting}

El procedimiento seguido es apagar la pantalla, hacer vibrar el terminal y crear una hebra y dormirla un tiempo determinado.

\section{Librerias externas usadas}
\begin{itemize}
	\item \textbf{Para detección de posición}: Sensor, SensorEvent, SensorEventListener, SensorManager
	\item \textbf{Para reproducir sonido}: MediaPlayer
	\item \textbf{Para la vibracion}: Vibrator
	\item \textbf{Material usado para saber como bloquear terminal:} \url{https://github.com/Joisar/LockScreenApp}
		
\end{itemize}
%\bibliography{citas} %archivo citas.bib que contiene las entradas 
\bibliographystyle{plain} % hay varias formas de citar

\end{document}